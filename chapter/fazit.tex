\chapter{Resume}

Future work:\\
-improve depth tracking value preciion through a more elaborate calibration seesion where the depth correcton algorith is further optimized
\\
-improve image reading source code to accept hgher framerates. The hardware capacbilities provide image reading frequences of up to 90fps, current bottleneck is the c++ implementaion which only provides 30 fps max.
\\
-When utilizng higher framerates, the multiproceesng approach will be viable to gain performance. Optimizaton work on the implementaion can be done.\\
-Image manipulaton on the cpu is rather costly, even at small image sizes. The raspberry Pi has a "small" gpu unit onboard, whch is in dle state when the device is run in headless mode. The gpu could then be utilzed to do heavy weight image calculation such as stereo rectification.\\
-Optimization pont would also be creating a mor optimized verson of the color thresholding algorithm where the threshoding for all of the thracking colors is done in one run on the current frame. It is also to be evaluated if ths problem falls can be expressed as a SIMD functon and would benefit from the processng on the gpu.
\\
The current prototype is still missing an graphcal user interface on the display side. Further communication of states on the slave Pi's to the master unit would also be beneficial.
\\
The habd data that is used for prototyping is still hardcoded and only manually configurable. A calibration procedure for the system as well as a suitable data format for representing and translating these values for the IK algortihm s still to be done.
\\
-Shrnk tubing showed to be a utilzable material for the usecase. Downsinde of the material is that the colors used are the largest spectrum of colors available on the market and geting outer colors is relatively hard. Other materal shuold be evaluated.
