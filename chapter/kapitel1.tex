%!TEX root = ../main.tex
\chapter{Kapitel 1}

\section{Section}
sadfsdf
\section{Section}
Beispiel für eine Tabelle:
\begin{table}[ht]
	\centering
	\caption[Kurztitel Tabelle]{Hier steht die lange \underline{\textbf{ÜBER-}}schrift für die Tabelle}

		\vspace{1.0em}
		\begin{tabular}{|l|r|}
\hline
Text & 12\% \\
\hline
Text & 34\% \\
\hline
Text & 56\% \\
\hline
Text & 78\% \\
\hline
Text & 90\% \\
\hline
		\end{tabular}
	\label{tab:tabelle}
\end{table}


\noindent{}Dies ist lediglich ein Beispiel. Je nach beabsichtigter Aussage, können Tabellen ganz unterschiedlich aussehen. Allerdings haben Tabellen eine \emph{Über}schrift, während es sich bei Abbildungen um \emph{Unter}schriften handelt.

\section{Section}

\section{Section}
siehe Tabelle \ref{tab:tabelle}

\subsection{Subsection}

\subsection{Subsection}