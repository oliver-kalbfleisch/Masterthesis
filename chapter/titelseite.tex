%!TEX root = ../main.tex
\begin{titlepage}

\begin{center}

% Logo der Technische Hochschule Köln
% Kann auch in dieser Form in Schwarz/Weiß ausgedruckt werden; Graustufen sollten der .tif Version entsprechen
\begin{figure}[!ht]
%	\centering
		\includegraphics[width=0.26\textwidth]{images/THlogoheader.pdf}
\end{figure}

\vspace{0.4cm}

%Deutscher Titel
\begin{rmfamily}
\begin{huge}
\textbf{Masterthesis Medientechnolgie}\\	
\end{huge}
\vspace{0.5cm}
\begin{LARGE}
mit einem eventuell\\ganz langen Untertitel\\
\end{LARGE}
\end{rmfamily}

\vspace{0.8cm}

%Englischer Titel
% \begin{rmfamily}
% \textbf{\LARGE Title in English}\\
% \large with a very\\long subtitle\\
% \normalsize
% \end{rmfamily}

% \vspace{1.2cm}

%ausgearbeitet von...
\begin{large}
vorgelegt von\\ 
\vspace{0.3cm}
\begin{LARGE}
Oliver Kalbfleisch \\
\end{LARGE}
\end{large}

\vspace{1.2cm}

%zur Erlangung des akademischen Grades...
%\begin{large}
%zur Erlangung des akademischen Grades\\
%\vspace{0.1cm}
%\textsc{Master of Science (Ma.Sc.)}\\ 
%\end{large}

\vspace{0.6cm}

%vorgelegt an der...
\begin{large}
vorgelegt an der\\ 
\vspace{0.2cm}
\begin{scshape}
Technischen Hochschule Köln\\
Campus Deutz\\
Fakultät für Informations-,\\
Medien- und Elektrotechnik\\

\end{scshape}
\end{large}

\vspace{0.6cm}

%im Studiengang...
\begin{large}
im Studiengang\\ 
\vspace{0.1cm}
\textsc{Medientechnologie}
\end{large}


\vspace{1.2cm}

%Autor der Bachelorarbeit und die Prüfer
\begin{tabular}{rl}
        Erster Prüfer:  &  Prof. Dr. Peter Silie\\
       					&  \small Technische Hochschule Köln \\[1.0em]
       Zweiter Prüfer:  &  Prof. Dr. Maria Musterprof\\
       					&  \small Technische Hochschule Köln\\
\end{tabular}

\vspace{1.2cm}

%Ort, Monat der Abgabe
\begin{large}
Köln, im August \the\year
\end{large}

\end{center}
\newpage
\setcounter{page}{1}
\pagenumbering{gobble}
\huge\textbf{Masterarbeit}\\\\
\large
Titel: Titel der Arbeit\\
\textbf{Gutachter}:\\
	Prof. Dr. rer. nat. Adam Mustermann (TH Köln)\\
	Dipl.-Ing. Eva Musterfrau (Media AG, Frankfurt)\\
Zusammenfassung: Was ein Abstract ist wird in der DIN Norm 1426 festgelegt: es ist ein Kurzreferat zur Inhaltsangabe. Die Definition des American National Standards Institute (ANSI) lautet: „An abstract is defined as an abbreviated accurate representation of the con-tents of a document". Es sollten 8-10 Zeilen Text folgen.\\
\textbf{Stichwörter}: Stichwort, Stichwort (hier sollten maximal 5 Stichwörter folgen)\\
\textbf{Sperrvermerk}: (optional) Die Einsicht in diese Arbeit ist bis zum TT. Monat JJJJ gesperrt.\\
\textbf{Datum}: TT. Monat JJJJ\\
\newpage
\huge \textbf{Masters Thesis}\\\\
\large
\textbf{Title}: Title of the work\\
\textbf{Reviewers}:\\
	Prof. Dr. rer. nat. Adam Mustermann (TH Köln)\\
	Dipl.-Ing. Eva Musterfrau (Media AG, Frankfurt)\\
\textbf{Abstract}: Hier sollte eine Übersetzung des obigen Abstracts auf Englisch erfolgen (und kein komplett neuer Text). Auch hier bitte die Begrenzung auf maximal 10 Zeilen Text ein-halten.\\
\textbf{Keywords}: (hier die Übersetzung der obigen Stichworte)\\
\textbf{Restriction notice}: (optional, falls Sperrvermerk oben ebenfalls existiert) The access to this thesis is restricted until <dd month YYYY>.\\
\textbf{Date}: dd month YYYY\\

\end{titlepage}
