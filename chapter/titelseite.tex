%!TEX root = ../Masterthesis.tex
\begin{titlepage}

\begin{center}
\begin{figure}[!ht]
	\centering
		\includegraphics[width=\textwidth]{images/th_color_bar.png}
\end{figure}
\end{center}
%Deutscher Titel
\begin{flushleft}
\begin{Large}
Masterthesis Medientechnolgie\\
\end{Large}
\vspace{0.5cm}
\begin{LARGE}
\textbf{RHOT-A real-time hand and object tracking system with low cost consumer grade hardware}
\end{LARGE}
\end{flushleft}
\vspace{1.0cm}
\begin{flushleft}
\begin{Large}
vorgelegt von\\ 
\vspace{0.3cm}
\begin{LARGE}
\textbf{Oliver Kalbfleisch} \\
\end{LARGE}
\end{Large}
\end{flushleft}
\vspace{2.0cm}
\begin{flushleft}
\begin{Large}
Erstgutachter: Prof. Dr. Arnulph Fuhrmann(TH Köln) \\[1.0em]
Zweitgutachter: Prof. Dr. Stefan Michael Grünvogel(TH Köln)
\end{Large}
\end{flushleft}
\vspace{1.5cm}
\begin{flushleft}
\begin{large}
Juni \the\year
\end{large}
\end{flushleft}
\begin{figure}[!ht]
\begin{flushright}
\includegraphics[width=0.25\textwidth]{images/TH_bottom_logo.png}
\end{flushright}
\end{figure}
\newpage
\setcounter{page}{1}
\pagenumbering{gobble}
\huge\textbf{Masterarbeit}\\\\
\large
Titel: RHOT A real time hand and object tracking system with low cost consumer grade\\
\textbf{Gutachter}:\\
	Prof. Dr. Arnulph Fuhrmann (TH Köln)\\
	Prof. Dr. Stefan Michael Grünvogel (TH Köln)\\
Zusammenfassung: Was ein Abstract ist wird in der DIN Norm 1426 festgelegt: es ist ein Kurzreferat zur Inhaltsangabe. Die Definition des American National Standards Institute (ANSI) lautet: „An abstract is defined as an abbreviated accurate representation of the con-tents of a document". Es sollten 8-10 Zeilen Text folgen.\\
\textbf{Stichwörter}: Stichwort, Stichwort (hier sollten maximal 5 Stichwörter folgen)\\
\textbf{Sperrvermerk}: (optional) Die Einsicht in diese Arbeit ist bis zum TT. Monat JJJJ gesperrt.\\
\textbf{Datum}: TT. Monat JJJJ\\
\newpage
\huge \textbf{Masters Thesis}\\\\
\large
\textbf{Title}: RHOT A real time hand and object tracking system with low cost consumer grade\\
\textbf{Reviewers}:\\
	Prof. Dr. Arnulph Fuhrmann (TH Köln)\\
	Prof. Dr. Stefan Michael Grünvogel (TH Köln)\\
\textbf{Abstract}: Hier sollte eine Übersetzung des obigen Abstracts auf Englisch erfolgen (und kein komplett neuer Text). Auch hier bitte die Begrenzung auf maximal 10 Zeilen Text ein-halten.\\
\textbf{Keywords}: (hier die Übersetzung der obigen Stichworte)\\
\textbf{Restriction notice}: (optional, falls Sperrvermerk oben ebenfalls existiert) The access to this thesis is restricted until <dd month YYYY>.\\
\textbf{Date}: dd month YYYY\\

\end{titlepage}
