%!TEX root = ../Masterthesis.tex
\begin{titlepage}

\begin{center}
\begin{figure}[!ht]
	\centering
		\includegraphics[width=\textwidth]{images/Th_color_bar.png}
\end{figure}
\end{center}
%Deutscher Titel
\begin{flushleft}
\begin{normalsize}
Masterthesis Medientechnolgie\\
\end{normalsize}
\vspace{0.5cm}
\begin{huge}
\textbf{A real-time hand and object tracking system with low cost consumer-grade hardware}
\end{huge}
\end{flushleft}
\vspace{1.0cm}
\begin{flushleft}
\begin{small}
vorgelegt von\\ 
\vspace{0.3cm}
\textbf{Oliver Kalbfleisch} \\
\end{small}
\end{flushleft}
\vspace{2.0cm}
\begin{flushleft}
\begin{small}
Erstgutachter: Prof. Dr. Arnulph Fuhrmann(TH Köln) \\[1.0em]
Zweitgutachter: Prof. Dr. Stefan Michael Grünvogel(TH Köln)
\end{small}
\end{flushleft}
\vspace{1.5cm}
\begin{flushleft}
\begin{small}
Juni \the\year
\end{small}
\end{flushleft}
\begin{figure}[!ht]
\begin{flushright}
\includegraphics[width=0.25\textwidth]{images/TH_bottom_logo.png}
\end{flushright}
\end{figure}
\newpage
\setcounter{page}{1}
\pagenumbering{gobble}
\LARGE\textbf{Masterarbeit}\\\\
\small
Titel: Ein echtzeit Hand- und Objektrackingsystem basierend auf kostengünstiger Hardwarekomponenten\\
\textbf{Gutachter}:\\
	- Prof. Dr. Arnulph Fuhrmann (TH Köln)\\
	- Prof. Dr. Stefan Michael Grünvogel (TH Köln)\\
\textbf{Zusammenfassung}:\\ 
Diese Abschlussarbeit beschäftigt sich mit der Frage der Umsetzbarkeit eines Hand- und Objekt-Tracking Systems, basierend auf günstigen consumer-grade Hardwarekomponenten. Um einen niedrigen Systempreis zu erzielen, wurden zwei Raspberry Pi Minicomputer mitsamt kompatiblen Kameras für die Bildaufnahme und Verarbeitung verwendet. Die aufgenommenen Daten werden an einen „Master“ PC kommuniziert, welcher sich um die Tiefen- und Positionsberechnung sowie das finale Rendering kümmert. Verschiedene Formen von Fingermarkierungen wurden im Zuge der Arbeit vorgestellt und in Bezug auf die Präzision des Trackings und der Einschränkung der haptischen Fähigkeiten der Finger mit der Markierung bewertet. Die finale Auswertung des entstandenen Prototyps zeigt, dass mit der ausgewählten Hardware grundsätzlich ein, mit gewissen Einschränkungen, funktionierendes Trackingsystem realisiert werden konnte.\\
\textbf{Stichwörter}: Echtzeit, Inverse Kinematik, Hand-tracking, color-tracking,Objekt-tracking, günstig\\
\textbf{Datum}: 29. Juni 2018\\\\\\
\LARGE \textbf{Masters Thesis}\\\\
\small
\textbf{Title}: A real-time hand and object tracking system with low cost consumer grade\\
\textbf{Reviewers}:\\
	- Prof. Dr. Arnulph Fuhrmann (TH Cologne)\\
	- Prof. Dr. Stefan Michael Grünvogel (TH Cologne)\\
\textbf{Abstract}: \\This paper will evaluate if it is possible to build a hand and object tracking system on low cost consumer grade hardware components.
For image acquisition and processing, a set of Raspberry Pi's with the matching camera is used to keep system cost low. These units are linked to a master unit which takes care of stereoscopic depth calculations, the position value processing and the final rendering of a hand and object model in digital space. Several forms of colored marker types are assessed for the prototype application. The final results show that it is possible to achieve a tracking system solution with the used hardware components that has a suitable tracking update frequency and precision.\\
\textbf{Keywords}: Hand-tracking, Object-tracking, inverse kineamtics, based on low-cost, realtime \\
\textbf{Date}: June 29th, 2018\\

\end{titlepage}
