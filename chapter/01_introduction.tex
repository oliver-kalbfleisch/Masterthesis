%!TEX root = ../Masterthesis.tex
\pagenumbering{arabic}
\chapter{Introduction}
The standard interface between human and computer has for long years been mouse and keyboard. But with the advance of technology, new interfacing methods were developed in the last few years.
Touch technology for interfacing with mobile devices and desktop computers has become a reliable technology and has been integrated into our everyday lives.
Advances in capabilities of CPU as well as GPU hardware has built a foundation for the use of advanced augmented and virtual reality technology. 3D and stereoscopic rendering can now be accomplished even on mobile devices(with some limitations). For an intensely immersive experience, VR goggles are used to explore digitally created worlds.\\
With this level of immersiveness, a touch device or just a mouse and keyboard setup is rather hindering for the user experience and the logical and more natural way of interfacing with our digital technology would be by utilizing the built-in human control elements that we carry around with us every day.\\ The human hands offer a form of control element that provides a large number of "\textit{Degrees-of-Freedom}" (DOF's) and furthermore adds the posibility of combining these witch a tactile component.The first more humble attemps of solving this problem already began in the early 1980's, where a lack of computational power and low-res imaging hardware made more complex systems impossible. These systems were mostly aimed at registering simple hand gerstures like pointing as an intefrace method and were not able to reconstruct full hand positions in realtime\cite{Bolt.1980}. Modern day elaborate system for hand tracking use infrared or stereoscopic Cameras to achieve a  more performant tracking result, utilizing the calculation capabilities of standard consumer computers. These systems are mostly specialized for the task they are doing and are therefore rather expensive to attain. \\
This thesis assesses, if it is possible to create a real-time hand and object racking system based on easily accessible consumer grade hardware and open source programs. The system should provide an experience for the user that is as natural as possible in terms of grabbing precision and the components should not hinder the motion capability of the hand. Furthermore the system should aim at being comfortable in terms of design as a cumbersome system would not be used  in everyday life.
Before evaluating a prototype system, an introduction to the anatomical properties of the human hand will be given.The understanding of these properties plays a major role in the setup for a hand simulation.As a follow-up, an overview of existing technologies in terms of tracking and display of the human hand will be given before the system conception, construction and analysis parts.

