%!TEX root = ../Masterthesis.tex
\pagenumbering{arabic}
\chapter{Introduction}
The standard interface between human and computer has for long years been mouse and keyboard. But with the advance of technology, new interfacing methods were developed in the last few years.
Touch technology for interfacing with mobile devices and desktop computers has become a reliable technology and has been integrated into our everyday lives.\\
Advances in capabilities of CPU as well as GPU hardware has build a foundation for the use of advanced AR and VR technology. 3D and stereoscopic rendering can now be accomplished even by mobile hardware (with some limitations) without on-board hardware. For an intensely immersive experience VR goggles are used to explore digitally created worlds.\\
With this level of immersivenes, a touch device or just a mouse and keyboard setup is rather hindering the user experience and the logical and more natural way of interfacing with our digital technology would be by utilizing the build in human control elements that we carry around with us every day. The human hands offer a build-in control element that provides a large number of "\textit{Degrees-of-Freedom}" (DOF's) and furthermore adds the posibility of combining these witch a tactile component.The first more humble attemps ofe solving this problem already began in the early 1980's where a lack of computational power and low-res imaging hardware made more complex systems impossible. These systems were mostly aimed at registering simple hand gerstures like pointing as an intefrace method an were not able to reconstruct full hand positions in realtime\cite{Bolt.1980}.Modern day elaborate system for hand tracking use infrared or stereoscopic Cameras to achieve a performant tracking result, utilizing the calculation capability of standard consumer computers. These systems are rather specialized for the task they are doing and are therefore rather expensive to attain. \\
This thesis assesses if it is possible to create a real-time hand and object racking system based on easily accessible consumer grade hardware and open source Programs as well as displaying the tracking result on a head mounted display. The system should provide an experience for the user that is as natural as possible in terms of grabbing precision and  the components should not hinder the motion capability of the hand. furthermore the system should aim at being comfortable in terms of design as a cumbersome system would not be used  in everyday life.
Before evaluating a prototype system, an overview of existing technologies in terms of tracking and display of the human hand will be given. A small overview over the anatomical properties of the human hand will be given as the understanding of these plays a major role int he setup for the hand simulation.

