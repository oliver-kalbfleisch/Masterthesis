%!TEX root = ../Masterthesis.tex
\chapter{Digital hand models}
After retrieving all needed data from the real world counterpart and calculation for pose estimation is finished, the returned solution has to be displayed in the digital world. Therefore a digital hand model is needed to represent the calulated data. Modern day computer animation techniques make it possible to create nearly photorealistic digital replicas of human skin wrapped onto anatomically precisely modeleld bodyparts. These highly complex modelsacquire a lot of calulation resources and are mostly not appropeiate for a realtime rendering approach.\\
When apdapting calcualtion data to digital hand models, those of lesser complexity are preffered, sacrificing the quality of the resulting output for speed and more fluent motion results.
