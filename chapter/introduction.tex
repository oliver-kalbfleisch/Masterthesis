%!TEX root = ../Masterthesis.tex
\pagenumbering{arabic}
\chapter{Introduction}
The standard interface between human and computer has for long years been mouse and keyboard. But with the advance of technology, new interfaceing methods were developed in the last few years.\\
Touch technology for interfacing with mobile devices and desktop computers has become a reliable technology and has been integrated into our everyday lives.
advances in capabilites of CPU as well as GPU hardware has build a foundation for the use of advanced AR and VR technology. 3D and stereoscopic rendering can now be accomplished even by mobile hardware (with some limitations) without the need of specifci hardware. For an intensly immersive experience VR googles are used to explore digitally created worlds.\\
With this level of immersivenes, a touch device or just a mouse and keyboard setup is rather hindering the user experience and the logical and more natural way of interfacing with our digital technology would be by utilizing the build in human control elements that we carry around with us every day. The human hands offer a build-in control element that provides a large number of Ddegrees-of-Freedom (DOF's) and furthermore adds the posiibiliy of combine these witch a tactile component.\\The first more humble attemps for the solving of this problem already began in the early 1980's where a lack of computational power and low-res imaging hardware made more complex systems impossible. These systems were mostly aimed at registering simple hand gerstures like pointing as an intefrace method an were not able to reconstruct full hand positions in realtime\cite{Bolt.1980}.\\\\This thesis aims to display a method of combining the tracking of a physical object and the human hand(s) and merging this data into the digital space. The system should be composed of mostly consumer grade technology and easily accesible hardware to provide a low cost solution. The resulting experience for the user should be as natural as possible therefore the hindering factor of the tracking solution in terms of motion capability for the hands should be kept as and as comfortable as possible.\\\\
To achieve such a solution,different systems for tracking of objects and bodyparts as well as the display of these in digital space will be presented and assessed. The results from the assessment wil be incorporated into a system prototype. 
\todo{more context for introduction?}
