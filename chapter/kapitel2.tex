%!TEX root = ../MAsterthesis.tex


\chapter{Description of hand in digital space}
Tracking of the human hand has always been a challenging Problem. In comaprison to other larger bodyparts like the Arm or the head, the human hand itself contains a large variety of smaller parts, namely bones and muscles. These components have to be taken into acount when trying to replicate the natural motion of the hand in digital space.\\
\section{Physiological structure of the human hand}
\citep{LEE.1995} describes the human hand as "an articulated structure with about 30 degrees of freedom [which] changes shape in various ways by its joint movemnents."
\begin{figure}[H]
	\includegraphics[scale=0.8]{images/hand.jpg}
	\label{Handstructure} 
	\caption{Bone structure of the human left hand (\cite{LEE.1995})}
\end{figure}
All of the hand components are connected to at least one neighboring component via a joint. Teh joints affect the position of the connected components. To describe the movement of the hand components, we can use the roation angles of the joints to correlate to a specific position.\\
To do so, we define a local coordinate system for each of the exiting hand joints. By doing so, we achieve a sequence of rotaions in the local coordinate systems of the joints. Such a sequence can then be used to describe a specific movement and/or position of a component.
Not all of the joints in the human hand have equal degrees of freedom. Their functionality can be classified in the amount of DOFs (Degrees of freedom)\cite{KOREIN.1985}
\begin{itemize}
\item 1 DOF \\
	- A joint movement that can perfom a \textbf{flexion} or \textbf{twist} in one direction
\item 2 DOF \\
	- A joint movement that can perform \textbf{flexion} in more than one direction (\textbf{directive})
\item 3 DOF\\
	- A joint movement that permits simultaneous \textbf{directive} and \textbf{twist} movements.(\textbf{spherical})
\end{itemize}
\begin{figure}[H]
\includegraphics[scale=0.8]{images/Hand_DOFs.JPG} 
\caption{Representation of the DOFs of the human hand}
\label{dof_image} 
\end{figure}
When looking at the DOFs diplayed in Figure \ref{dof_image}, each finger (II-V) sums up to 4 DOFs and the thumb to 5 DOFs. Also considering 6 DOFs for the rotation and position of the whole and itsself, the result gets us to 27 DOFs  for the human hand.
\subsection{Constraints in Hand Motion}
A full usage of all the declared DOFs would lead to al large amount of possible combination. Since the hand is not only made up of bones but also Muscles and the skin, we can impose some constraints( \cite{Badler.1987}) to the movement of the joints. Ling, Wu and Huang(\cite{LIN.2000}) propsed following classification for the constrainsts:
\begin{itemize}
\item \textbf{Type I constraints}\\
	-A constraint that limits the range of finger motions based on hand anatomy
	\item \textbf{Type II constraints}\\
	- A constraint that the position of the joints during finger movement
	\item \textbf{Type III constraints\\}
	-A constraint that limits position based on natural hand motions
\end{itemize} 
The \textbf{Type I} and \textbf{Type II} constarints rely on the physiological and mechanical properties of the humand.\textbf{Type III} constraints are results of common and natural
movements and can be differing form person to person. As these movements are to some degree simular for everyone, a broad grouping can be applied. The curling of the fingers at the sane time when forming a fist is way more natural than curling each finger by itsself. Here the motion of the hand is quite simular between different persons, but the constraints cannot be described in a mathematical form. \\
 A \textbf{Type I} constraint example would be that the position of the fingertip is kimited by the length of the other finger segments and therby can only reach as far as the combined length.\\An example for \textbf{Type II} constraints would be that, for your fingertip to touch your hand palm, all joints in the finger have to be bend to achieve this position.
The following inequalities can be used to describe these constraints:\\
\textbf{Type I}
\begin{equation}
\begin{split}
0°\leq \Theta _{MP\_flex} \leq 90°\\
0°\leq \Theta _{PIP\_flex} \leq 110°\\
0°\leq \Theta _{DIP\_flex} \leq 90°\\
-15°\leq \Theta _{MP\_abduct/adduct} \leq 15°
\end{split}
\end{equation}
A further constraint that is specific to the middle finger is, that this fingers MP normally does not abduct and adduct much. Therefore we can infer an approximation and thereby remove 1DOF from the model:
\begin{equation}
\Theta _{MP\_abduct/adduct}=0°
\end{equation}
The same behavior can be seen in the combination of hand parts labeled W(the connection point between hand and lower arm). This approximation allso eliminates one DOF on the connected thumb:
\begin{equation}
\Theta _{W\_abduct/adduct}=0°
\end{equation}
Since the DIP,PIP and MP jonts of our index, middle, ring, and little fingers only have 1DOF for flexion, we can further asume that their motion is limited to movement in one plane. \\
\textbf{Type II}\\
The \textbf{Type II} constraints can be split into interfinger and intrafinger constraints. Regarding intrafinger constraints between the joints of the same finger, human hand anatomy implies that to bend the DIP joints  on  either the index, middle, ring or little fingers,the corresponding PIP joints of thath finger must also be bent. The approximation for this relation[\cite{Rijpkema.1991}] can be described as :

\begin{equation}
\Theta _{DIP} =\frac{2}{3}\Theta _{PIP}
\end{equation}
Interfinger constraints can be imposed between joints of adjacent fingers. Interfinger constraints describe that the bending of an MP joint in the index finger forces the MP joint in the middle finger to bend as well.\\
 When combinig the constraints described in the above equations, the starting number 21 DOF's of the human hand can be reduced to 15. Inequalities for these cases, obtained through empiric studies, can be found in \citep{LEE.1995}.\\
\section{Kinematics}
 The preceeding sections gave an overview of how we can describe a model of the human hand and introduced some limiting constraints. With the model and the constraints, we can now start to build a kinematic system for the animation of the model.\\
Kinematic systems contain so called \textit{kinematic chains}, which consist of a \textit{starting point}or \textit{root}, kinematic elements like \textit{joints}, \textit{links} and an \textit{endpoint}, also called \textit{end effector}. Applied to the human hand, the whole hand model represents the kinematic system. This system contains several \textit{kinematic chains}, namely the fingers of the hand with the fingertips beeing the \textit{end effectors} of each of these chains.\\
As we begin to move our hands, the states of the kinemtic chains begin to change. Joint angles and end effector positions are modified until the end position is reached. To represent the new position and agnle dataset of our physical hand with our kinematic system, two major paths for achieving a solution can be taken.
 \subsection{Forward Kinematics}Forward
 \label{Forward Kinematics}  Kinmatics (FK) uses the knowledge of the new angles and positions after the application of known transformations to the kinematic chain. The data of the \textit{joints} and \textit{links} between the \textit{root} and the \textit{end effector} is then used to solve the problem of finding the \textit{end effector's} position. The advantage of an FK solution is that there is always an unique solution to the problem. In consequence, this approach is commonly used in the field of robotics, where the information on the chain elements is easily available.\\
The tracking of the human hand and all if its chain components is rather complicated.Therefore a solution which takes a known position of the \textit{end effektor} and calculates the parameters for the rest of the cain would be more desireable.
 \subsection{Inverse Kinematics}
\begin{quote}Inverse Kinematics (IK) is a method for computing the posture via estimating each individual degree of freedom in order to satisfy a given task~\cite[S.~14]{AndreasAristidouandJoanLasenby.}\end{quote}
The concept of \textit{Inverse Kinematics}  (IK)already describes it's principle in it's name. It take the reversed approach in comparison to the FK principle in chapter \ref{Forward Kinematics}. Instead of knowing the states of the chain elements and calculating the resulting position of the \textit{end effector}, we take the position of the \textit{end effector} and try to retrieve the possible states of the other chain elements. In contrary to having a unique solution with the FK approach, the IK approach can end at the point of not finding a suitable solution. Figure \ref{IkSolutions} diplays three possible outcomes for the IK approach.
\begin{figure}[H]
\includegraphics[scale=0.6]{images/Ik_figure.jpg}
\caption{Possible solution for an IK problem of a human finger:\\(a)The given target position of the end effector can not be reached. (b) The given target can only be reached by one solution.(c) The target position can be reached with multiple different solutions.}
\label{IkSolutions}
\end{figure}


\section{Digital hand models}